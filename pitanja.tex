% !TEX encoding = UTF-8 Unicode

\documentclass[a4paper]{article}

\usepackage[utf8x,utf8]{inputenc}
\usepackage[T1]{fontenc}
\usepackage{amssymb}
\usepackage{amsmath}
\usepackage{hyperref}

\hypersetup{
    colorlinks=true,
    linkcolor=blue,
    filecolor=magenta,
    urlcolor=cyan,
}

\urlstyle{same}

\begin{document}

\title{Istrazivanje Podataka}

\author{Daniel Doža}

\maketitle

\section{Sta je istrazivanje podataka?}

Ne postoji precizna definicija istrazivanja podataka, ali se smatra da je to skup algoritama i
tehnika koji omogucava automatsko zakljucivanje nekih cinjenica i veza iz velikog skupa podataka,
koji bi mozda inace ostali neotkriveni.

\section{Koji su zadaci istrazivanja podataka?}

Zadaci u istrazivanju podataka se mogu podeliti u dve grupe: zadaci opisivanja i zadaci
predvidjanja. Zadaci predvidjanja bave se odredjivanjem vrednosti nekog atributa na osnovu vrednosti
neki drugih atributa. Ove atribute nazivamo \emph{zavisne promenljive} ili \emph{ciljne promenljive}
a atribute na osnovu kojih dolazimo do zakljucaka \emph{opisne promenljive} ili \emph{nezavisne
promenljive}.

\begin{enumerate}
    \item \textbf{Klasifikacija} pravimo funkciju od nezavisnih varijabli da bi dobili zavisne
        varijable
    \begin{itemize}
        \item klasifikacija - podatke, na primer: da li ce kupac kupiti neki artikal
        \item regresija - za kontinualne podatke, na primer: koliko ce stepeni biti u neko vreme
    \end{itemize}
    \item \textbf{Prepoznavanje obrazaca} - nalazimo veze u podacima: na primer ako je kupac kupio
        pelene, ver
    \item \textbf{Klasterovanje} - grupisemo podatke prema slicnosti, na primer: grupisemo dokumente
        vezane za ekonomiju u jedan klaster, a dokumente vezane za medicinu u drugi klaster.
    \item \textbf{Otkrivanje anomalija} - otkrivanje podataka koji su veoma drugaciji od ostalih.
        Ove podatke nazivamo anomalije ili autlajeri
\end{enumerate}

\section{Sta su podaci?}

Skup podataka mozemo predstaviti kao skup \emph{objekata}. Ovi objekti se takodje mogu nazivati i
\emph{slog}, \emph{slucaj}, \emph{uzorak}, \emph{vektor}\ldots Objekte opisuju njihovi
\emph{atributi} koji nam govore o osobinama tog objekta, na primer: boja, visina, tezina\ldots
Atributi se jos nazivaju i \emph{karakteristika}, \emph{varijabla}, \emph{polje}\ldots

\section{Sta je atribut?}

Atribut je karakteristika objekta. Atribut moze da varira izmedju razlicitih objekata ili da varira
u vremenu. Boja ociju je razlicita za dva coveka, a na primer, temperatura nekog predmeta varira u
toku vremena.

\section{Sta je vrednost atributa?}

Vrednost atributa je broj ili simbol koji je pridruzen atributu. Moramo razlikovati vrednost
atributa od samog atributa, jer vrednosti mogu da imaju neke osobine koje atribut nema, i obrnuto.
Na primer ako za zaposlenog cuvamo identifikacioni broj i broj godina, ima smisla racunati prosek
godina, dok nema smisla racunati prosek identifikacionog broja. Jedina logicna operacija sa
identifikacionim brojevima je poredjenje jednakosti.

\section{Kako odredjujemo tip atributa?}

Tip atributa mozemo odrediti na osnovu broja vrednosti koji moze da sadrzi:

\begin{itemize}
    \item \textbf{diskretni} (konacni, ili prebrojivo beskonacni skupovi vrednosti, primer:
        postanski brojevi)
    \item \textbf{neprekidni} (realni brojevi, primer: temperatura, tezina, pritisak, brzina)
\end{itemize}

\emph{Asimetricni atributi} - kod njih je bitna samo ne-nula vrednost. Na
primer vrednost 1 ako je student pohadjao kurs, a vrednost 0 ako nije. U tom slucaju bi nas zanimali
samo studenti sa vrednoscu 1.

Takodje, tip atributa mozemo posmatrati na osnovu operacija koje se mogu izvrsiti nad njihovim
vrednostima

\begin{center}\begin{tabular}{llll}
    \hline
    Vrsta Operacije & Rbr & Operacija & Tip Atributa \\ \hline
    razlicitost & 1 & \(=\) i \(\neq\)& Imenski(1) \\
    uredjenje & 2 & \(<\), \(>\), \(\leqslant\), \(\geqslant\) & Redni(1,2) \\
    aditivnost & 3 & \(+\), \(-\) & Intervalni(1,2,3) \\
    multiplikativnost & 4 & \(\times\), \(\div\) & Razmerni(1,2,3,4) \\ \hline
\end{tabular}\end{center} 

\begin{itemize}
    \item \textbf{Kategoricki} - imenski i redni
    \item \textbf{Numericki} - intervalni i razmerni
\end{itemize}

\section{Koje su karakteristike skupa podataka?}

\begin{enumerate}
    \item dimenzionalnost - predstavlja broj atributa koje objekti imaju
    \item retkost - uzmimo na primer asimetricne podatke, mozda razmatramo samo 1\% od ukupnog broja
        objekata
    \item rezolucija - sa kojim nivoom detalja gledamo na podatke. Na primer zemlja gledana sa par
        metara je proprilicno neravna, a gledano sa par desetina kilometara je poprilicno glatka.
\end{enumerate}

\section{Koji su tipovi skupa podataka?}

Tipovi podataka nisu precizno definisani, ali ih mozemo grupisati u tri grube kategorije:

\begin{enumerate}
    \item Slogovi
    \item Grafovski podaci
    \item Uredjeni podaci
\end{enumerate}

\section{Sta su slogovi?}

Podrazumeva da su podaci organizovani kao torke koje predstavljaju objekte. Svaki objekat ima fiksan
broj atributa. U svojoj osnovnoj formi, smatramo da ne postoje veze izmedju torki, kao ni izmedju
atributa, i svaki objekat ima isti skup atributa.

\paragraph{Transakcije}

Transakcija je specijalan slucaj torki. Asocijacija se moze napraviti sa \emph{potrosackom korpom}.
Svaka torka sadrzi skup elemenata koje je kupac kupio u jednoj kupovini.

\paragraph{Matrice podataka}

Ukoliko svi objekti imaju isti broj atributa, podaci se mogu posmatrati kao matrica. Redovi ove
matrice predstavljaju objekte, a kolone predstavljaju atribute (obrnuto je takodje dozvoljeno). U
ovakvom uredjenju, objekte mozemo posmatrati kao \(n\)-dimenizione vektore, gde su
dimenzije odredjene atributima.

\section{Sta su grafovski podaci?}

\paragraph{Grafovi sa vezama izmedju podataka u granama}

Objekti su u ovom modeli predstavljeni kao cvorovi grafa, dok su veze izmedju objekata prikazane
granama. Kao primer mozemo uzeti Web stranice.

\paragraph{Objekti predstavljeni kao grafovi}

Ako su objekti sacinjeni od podobjekata koji imaju medjusobne veze, takve objekte cesto
predstavljamo kao grafove. Kao primer mozemo uzeti hemijska jedinjenja, gde cvorovi predstavljaju
atome, a grane predstavljaju hemijske veze.

\section{Sa su uredjeni podaci?}

Ponekad su podaci uredjeni na osnovu vremena i/ili prostora.

\paragraph{Vremenski podaci}

Uredjeni su kao slogovi, s tim sto je svakom slogu pridruzeno vreme. Ovim mozemo odredjivati na
primer porast u kupovini slatkisa pred noc vestica.

\paragraph{Podaci u odredjenom redosledu}

Slicni su vremenskim podacima, ali nemaju vremenske odrednice, vec su poredjani u uredjenom
rasporedu. Primer su genetske sekvence.

\paragraph{Serijski podaci}

Vrlo slicni vremenskim podacima, samo sto svaki objekat cini serija podataka izmerenih u toku nekog
perioda.

\paragraph{Prostorni podaci}

Objekti sadrze prostorne odrednice. Primer bi bio podaci o vremenskoj prognozi.

\section{Sta je prepoznavanje obrazaca?}

Prepoznavanje obrazaca se moze, u osnovnoj formi predstaviti preko binarne matrice. Moze se
posmatrati takva matrica da kolone predstavljaju artikle, a redovi transakcije kupaca. Polje
\((i,j)\) je jednako 1 ako je kupac kupio artikal, inace je 0. Cilj nam je da prepoznamo da li postoje
neka pravila u kupovini, tojest, da li su vece sanse da kupac kupi stvar \(x\) ako je kupio stvar
\(y\). Formalno: neka je data binarna matrica \(n \times d\), posmatramo podskupove kolona, takve
da sve imaju vrednost 1. Svakom tom podskupu se dodeljuje podrska \(s\), koja predstavlja
ucestalost ponavljanja tog podskupa u odnosu na ceo skup. Ukoliko je \(s\) vece od minimalne
podrske, smatramo da je obrazac cest.

\section{Sta je podrska pravila pridruzivanja?}

Neka su \(A\) i \(B\) dva skupa obrazaca. Podrska \(sup(A \implies B)\) je
definfinisana kao \(\#(A \cup B) / N\), Gde \(\#(A \cup B)\) predstavlja broj
zadovoljavajucih obrazaca, a \(N\) ukupan broj redova u kompletnom skupu.

\section{Sta je pouzdanost pravila pridruzivanja?}

Neka su A i B dva skupa obrazaca. Pouzdanost \(conf(A \implies B)\) je definisana kao
\(\#(A \cup B) / \#(A)\).

\section{Sta je klasterovanje? (gruba definicija)}

Klasterovanje je grupisanje objekata na osnovu njihove `slicnosti'. Uvidja se da je od velike
vaznosti dizajn funkcije `slicnosti'.

\section{Sta je otkrivanje elemenata van granica? (gruba definicija)}

Zadatak je otkriti element koji je u velikoj meri razlicit od svih ostalih. Ovaj element zovemo
anomalija ili autlajer. Mozemo autlajere posmatrati i kao `element koji je toliko razlicit od
ostalih, da se moze posumnjati da je nastao nekim razlicitim mehanizmom'. Primeri bi bili upad u
racunarski sistem, zloupotreba kreditnih kartica\ldots

\section{Sta je klasifikacija? (gruba definicija)}

Klasifikacija je problem odredjivanja vrednosti nekog specijalnog atributa. Problem klasifikacije je
problem nadgledanog ucenja. Formiramo skup podataka koje nazivamo  podaci za trening, na osnovu
kojih nas algoritam odredjuje odnos ostalih atributa i specijalnog atributa koji se trazi. Nakon
toga test podaci se koriste da se utvrdi preciznost algoritma i eventualno podese parametri u cilju
povecanja preciznosti. Onda mozemo koristiti dobijen algoritam za odredjivanje specijalnog atributa
u skupovima podataka gde je on nepoznat.

\section{Navesti par primera istrazivanja podataka}

\begin{enumerate}
    \item rasporedjivanje artikala u prodavnici
    \item prepouke kupcima
    \item anomalije u logovima aplikacija
\end{enumerate}

\section{Sta je slicnost/razlicitost objekata, obrazaca i atributa?}

Slicnost i razlicitost izmedju objekata su funkcije koje nam govore o tome koliko su dva objekta
slicna ili razlicita. Vece vrednosti funkcije slicnosti nam govore da su objekti vise slicni, a
obrnuto vazi za funkcije razlicitosti.  Uglavnom se funkcije slicnosti mere u vrednostima na
intervalu [0, 1], dok se funkcije razlicitosti mere na intervalu od 0 (objekti su isti) na vise.
Koriste se i termini rastojanje (distance), blizina (proximity).

Funkcije slicnosti i razlicitosti su od velikog znacaja, jer imaju uticaj na svaki problem u
istrazivanju podataka. Los izbor funkcije slicnosti moze da ima presudnu vrednost u tome da li smo
odradili dobar posao. Ova cinjenica nam govori da ne smemo zapostaviti izbor funkcije slicnosti i
samo se fokusirati na algoritamski deo problema istrazivanja podataka.

\section{Navesti primer funkcije slicnosti/razlicitosti za nominalne atribute \(p\) i \(q\).}

\begin{itemize}
\item \textbf{Slicnost}

\[sim(p, q) = 1 \iff p = q\]
\[sim(p, q) = 0 \iff p \neq q\]

\item \textbf{Razlicitost}

\[dist(p, q) = 0 \iff p = q\]
\[dist(p, q) = 1 \iff p \neq q\]

\end{itemize}

\section{Navesti primer funkcije slicnosti/razlicitosti za redne atribute \(p\) i \(q\)}.

Ako \(p\) i \(q\) mogu imati \(n\) razlicitih vrednosti, onda funkcije slicnosti i
razlicitosti definisemo na sledeci nacin:

\begin{itemize}
\item \textbf{Slicnost}

\[sim(p, q) = 1 - \dfrac{|p - q|}{n - 1}\]

\item \textbf{Razlicitost}

\[dist(p, q) = \dfrac{|p - q|}{n - 1}\]
\end{itemize}

\section{Navesti primer funkcije slicnosti/razlicitosti za intervalne i razmerne atribute \(p\) i
\(q\).}

\begin{itemize}
\item \textbf{Slicnost}

\[sim(p, q) = -dist(p,q)\]
\[sim(p, q) = \dfrac{1}{1 + dist(p, q)}\]

\item \textbf{Razlicitost}

\[dist(p, q) = |p - q|\]
\end{itemize}

\section{Sta treba da vazi za funkciju rastojanja \(d\) da bi ona bila metrika?}

Da bi funkcija rastojanja \(d\) je metrika ako i samo ako vazi:

\begin{enumerate}
    \item Pozitivna odredjenost
        \[d(p, q) \geq 0, \forall p, q\]
        \[d(p, q) = 0, \iff p = q\]
    \item Simetrija
        \[d(p, q) = d(q, p)\]
    \item Nejednakost trougla
        \[d(p, q) \leq d(p, r) + d(r, q)\]
\end{enumerate}

\section{Sta treba da vazi za funkciju rastojanja \(d\) da bi ona bila ultrametrika?}

Funkcija \(d\) je \emph{ultrametrika} ako je metrika i ako vazi:

\[d(p, q) \leq max\{d(p, z), d(z, q)\}, \forall p, q, z\]

\section{Koje se mere rastojanja cesto koriste za kvantitativne podatke?}

Hamingovo rastojanje, rastojanje Minkovskog, Mahalanobisovo rastojanje.

\section{Sta je rastojanje Minkovskog (prednosti/nedostaci)?}

Za dva objekta \(\overline{X}=(x_1,x_2,\ldots,x_n)\) i \(\overline{Y}=(y_1,y_2,\ldots,y_n)\)
Rastojanje minkovskog se definise kao:

\[\sum_{i=1}^{n} {({|x_i - y_i|}^p)}^{1/p}\]

Rastojanje Minkovskog za \(p = 2\) je Euklidsko rastojanje, za \(p = 1\) je Menhetn
rastojanje. Prednost ove metode je u tome sto je veoma intuitivna. Medjutim, to sto je intuitivna,
ne znaci da daje dobre rezultate, pogotovo u slucajevima velike dimenzionalnosti. Na primer ne uzima
u obzir koliko je neki atribut bitan za odredjivanje slicnosti. Takodje lose radi ako je nepoznata
raspodela\ldots

\section{Sta je Mahalanobisovo rastojanje (prednosti/nedostaci)?}

Jedan od nedostataka rastojanaj Minkovskog je sto zavisi samo od objekata nad kojim se formula
izracunava, a ne obraca paznju na distribuciju ostalih podataka. Mahalanobisovo rastojanje uzima u
obzir raspodelu podataka koristeci matricu kovarijansi. Neka su
\(\overline{X}=(x_1,x_2,\ldots,x_n)\) i \(\overline{Y}=(y_1,y_2,\ldots,y_n)\) dva objekta.
Mahalanobisovo rastojanje izmedju \(\overline{X}\) i \(\overline{Y}\) je:

\[
    Maha(\overline{X}, \overline{Y}) =
    \sqrt{
        (\overline{X} - \overline{Y}) \times \Sigma^{-1} \times (\overline{X} - \overline{Y})^{T}
    }
\]

Drugim recima uzimamo razliku vektora \(\overline{X}\) i \(\overline{Y}\) pomnozimo sa inverzom
matrice kovarijansi \(\Sigma\) i transponovanom razlikom vektora \(\overline{X}\) i \(\overline{Y}\).

\section{Kako se moze definisati slicnost podataka sa kategorickim atributima?}

Kad radimo sa kategorickim funkcijama obicno se vise koriste funkcije slicnosti nego razlicitosti
jer je se diskretne vrednosti mogu prirodnije porediti.  Slicnost podataka sa kategorickim
atributima se moze definisati preko slicnosti njihovih pojedinacnih atributa. Neka su
\(\overline{X}=(x_1,x_2,\ldots,x_n)\) i \(\overline{Y}=(y_1,y_2,\ldots,y_n)\) objekti. Njihovu
slicnost mozemo definisati kao:

\[
    Sim(\overline{X}, \overline{Y}) = \sum_{i = 1}^{n} S(x_i, y_i)
\]


Odavde vidimo da izbor funkcije \(S\) odredjuje citavu funkciju slicnosti. U najjednostavnijem
slucaju funkcija \(S\) se moze definisati kao

\[
    S(x_i, y_i) =
    \begin{cases}
        1 & x_i = y_i \\
        0 & x_i \neq y_i \\
    \end{cases}
\]

Medjutim, mozemo uvideti da je problem kod ove funkcije da ona ne uzima u obzir frekvenciju
razlicitih atributa. Uzmimo na primer atribut koji moze da ima vrednosti `Normalno', `Rak' i
`Dijabetes'. Najverovatnije je da ce 99\% podataka imati vrednost `Normalno' ali oni nece biti od
statisticke vaznosti isto toliko koliko i objekti sa vrednostima `Rak' i `Dijabetes'. Drugim recima,
velika vecina nam ne odredjuje dovoljno dobro slicnost izmedju objekata. Sa ovim na umu treba
kreirati nesto slicno Mahalanobisovom pristupu. Takav pristup naziva se \emph{inverzna frekvencija
ponavljanja}.

 Neka je \(p_i(x)\) broj slogova ciji \(i\)-ti atribut ima vrednost \(x\). Tada mozemo nasu funkciju
 \(S\) definisati kao:

 \[
    S(x_i, y_i) =
    \begin{cases}
        \dfrac{1}{p_i(x_i)^2} & x_i = y_i \\
        0 & x_i \neq y_i \\
    \end{cases}
 \]

\section{Kako se odredjuje slicnost tekstualnih dokumenata?}
\label{pitanje:slicnost_tekst_dokumenata}

Tekstualne dokumente mozemo smatrati multidimenzionim podacima kada bi ih posmatrali kao `vrece
reci'. To bi znacilo da bi kompletan set atributa nekog dokumenta bio ceo leksikon reci, a vrednosti
bi bile broj pojavljivanja odgovarajuce reci u dokumentu. Ovakav format bi znacio da ce vecina
atributa imati vrednost 0, sto bi dalje povlacilo da kada bismo koristili nesto kao sto je
rastojanje Minkovskog, dva slicna dugacka teksta ce uvek biti vise razlicita nego dva zapravo
razlicita kraca teksta. Da bismo ovo izbegli, koristimo kosinusno rastojanje. Neka su
\(\overline{X}=(x_1,x_2,\ldots,x_n)\) i \(\overline{Y}=(y_1,y_2,\ldots,y_n)\) objekti. Kosinusno
rastojanje definisemo kao:

\[
    \cos(\overline{X}, \overline{Y}) =
    \dfrac{\sum_{i=1}^{n} x_i \cdot y_i}{\sqrt{\sum_{i=1}^{n} x_i^2} \cdot \sqrt{\sum_{i=1}^{n} y_i^2}}
\]

Ova metoda ne uzima u obzir odnos pojavljivanja reci. Mi znamo da su tekstovi slicniji ukoliko se
podudaraju na recima koje se retko javljaju, nego na onim koje se cesto javljaju. \emph{Inverznu
frekvenciju dokumenta} definisemo kao:
\[id_i = \log(n/n_i)\]
gde je \(n\) ukupan broj dokumenata, a \(n_i\) je broj dokumenata u kojima se \(i\)-ta rec
pojavljuje.

Kako pojavljivanje jedne reci ne bi trebalo da poremeti celu meru, koristimo funkcije:

\[f(x_i) = \sqrt{x_i}\]
\[f(x_i) = \log(x_i)\]

\emph{Normalizovane} funkcije frekvencije reci se onda definisu kao

\[h(x_i) = f(x_i) \cdot id_i\]

Sada mozemo napisati kosinusnu meru slicnosti koristeci ove normalizovane funkcije frekvencije
ponavljanja reci

\[
    \cos(\overline{X}, \overline{Y}) =
    \dfrac{\sum_{i=1}^{n} h(x_i) \cdot h(y_i)}{\sqrt{\sum_{i=1}^{n} h(x_i)^2} \cdot
        \sqrt{\sum_{i=1}^{n} h(y_i)^2}}
\]


\section{Sta je rastojanje Minkovskog sa tezinama?}

U nekim slucajevima, nisu svi atributi objekta podjednako bitni u odredjivanju slicnosti. Na primer,
visina plate igra mnogo vecu ulogu nego pol u slucaju odobravanja kredita. U ovakvom slucaju mozemo
koristiti rastojanje Minkovskog sa tezinama (generalizovano rastojanje Minkovskog).

\[
    Sim(\overline{X}, \overline{Y}) = \left(\sum_{i=1}^{n} a_i \cdot |x_i - y_i|^p\right)^{1/p}
\]

Vrednost \(a_i\) nam govori o vaznosti \(i\)-tog atributa u poredjenju dva
objekta. Vrednosti \(a_i\) se dobijaju heuristickim metodama i u velikoj meri zavise od iskustva
analiticara.

\section{Kako se odredjuje slicnost dva sloga sa kvantitativnim i kategorickim atributima?}

Poprilicno jednostavan pristup resavanju ovog problema je da dodamo tezine dobijene za numericke
parametre i tezine dobijene za kategoricke parametre. Neka su
\(\overline{X}=(\overline{X_n}, \overline{X_c})\) i \(\overline{Y}=(\overline{Y_n}, \overline{Y_c})\)
objekti koji sadrze numericke i kategoricke atribute.

\[
    Sim(\overline{X}, \overline{Y}) =
        \lambda \cdot NumSim(\overline{X_n}, \overline{Y_n})
        + (1 - \lambda) \cdot CatSim(\overline{X_c}, \overline{Y_c})
\]

Ovde \(\lambda\) oznacava bitnost kategorickog i numerickog dela u izracunavanju slicnosti. Izbor
\(\lambda\) je tezak, pogotovo sa ogranicenim poznavanjem domena podataka. Generalno se uzima udeo
numerckih atributa u ukupnom skupu atributa, mada ovo ne znaci da je to dobar izbor.

\section{Sta je SMC (simple matching coefficient)?}
SMC (jednostavno uparivanje koeficijenata) se koristi za objekte sa binarnim atributima. Neka su
\(\overline{X}=(x_1,x_2,\ldots,x_n)\) i \(\overline{Y}=(y_1,y_2,\ldots,y_n)\) objekti sa binarnim
atributima. Definisemo
\begin{itemize}
    \item \(M_{01}\) --- broj atributa koji su jednaki 0 u \(\overline{X}\) i 1 u \(\overline{Y}\)
    \item \(M_{10}\) --- broj atributa koji su jednaki 1 u \(\overline{X}\) i 0 u \(\overline{Y}\)
    \item \(M_{11}\) --- broj atributa koji su jednaki 1 u \(\overline{X}\) i 1 u \(\overline{Y}\)
    \item \(M_{00}\) --- broj atributa koji su jednaki 0 u \(\overline{X}\) i 0 u \(\overline{Y}\)
\end{itemize}
Tada je
\[SMC(\overline{X}, \overline{Y}) = \dfrac{M_{11} + M_{00}}{M_{00} + M_{11} + M_{10} + M_{01}} \]

\section{Sta su Zakardovi koeficijenti? Kada se koriste?}
Zakardovi koeficijenti se koriste u slucaju objekata sa asimetricnim binarnim atributima. Definise
se kao kolicnik broja parova gde su obe vrednosti ne-nula i broja parova gde nisu obe vrednosti
nula. Neka su \(\overline{X}=(x_1,x_2,\ldots,x_n)\) i \(\overline{Y}=(y_1,y_2,\ldots,y_n)\) objekti
sa asimetricnim binarnim atributima. Definisemo
\begin{itemize}
    \item \(M_{01}\) --- broj atributa koji su jednaki 0 u \(\overline{X}\) i 1 u \(\overline{Y}\)
    \item \(M_{10}\) --- broj atributa koji su jednaki 1 u \(\overline{X}\) i 0 u \(\overline{Y}\)
    \item \(M_{11}\) --- broj atributa koji su jednaki 1 u \(\overline{X}\) i 1 u \(\overline{Y}\)
    \item \(M_{00}\) --- broj atributa koji su jednaki 0 u \(\overline{X}\) i 0 u \(\overline{Y}\)
\end{itemize}
Tada zakardove koeficijente izracunavamo
\[Jacc(\overline{X}, \overline{Y}) = \dfrac{M_{11}}{M_{11} + M_{10} + M_{01}}\]

\section{Sta su prosireni Zakardovi koeficijenti (koeficijenti Tanimoto-a)?}
\emph{Prosireni Zakardovi koeficijenti} se koriste u slucaju kada atributi nisu binarni, ali se
svodi na slucaj \emph{Zakardovih koeficijenata} ukoliko jesu.
\[
    ExtJacc(\overline{X},\overline{Y}) =
    \dfrac{
        \overline{X} \bullet \overline{Y}
    }{
        ||\overline{Y}||^2 + ||\overline{Y}||^2
        - \overline{X} \bullet \overline{Y}
    }
\]

\section{Kako se definise kosinusna slicnost dva objekta? Kada se koristi?}

Pitanje~[\ref{pitanje:slicnost_tekst_dokumenata}]

\section{Sta je korelacija dva objekta?}
Korelacija izmedju dva objekta koji imaju binarne ili neprekidne atribute je mera linearne
zavisnosti izmedju tih atributa.
\[
    \rho_{xy} = \dfrac{cov_{xy}}{\sigma_x \cdot \sigma_y}
\]
Kovarijansa
\[
    cov_{xy} = \dfrac{1}{n-1}\sum_{i=1}^{n} (x_i - \overline{x})(y_i - \overline{y})
\]
Standardna devijacija
\[
    \sigma_x = \sqrt{\dfrac{1}{n-1}\sum_{i=1}^{n} (x_i - \overline{x})^2}
\]
Srednja vrednost
\[
    \overline{x} = \dfrac{1}{n}\sum_{i=1}^{n} x_i
\]

Korelacija je uvek u intervalu \([-1,1]\). Korelaciju vrednosti 1 zovemo perfektna pozitivna
korelacija. Slicno je za negativnu perfektnu korelaciju.


\section{Kako izrazavamo slicnost diskretnih podataka?}
Slicnost diskretnih podataka mozemo predstaviti kao cenu transformacije jednog objekta u drugi.
Za prvih \(i\) atributa iz \(\overline{X}=(x_1,x_2,\ldots,x_n)\) i prvih \(j\) atributa iz
\(\overline{Y}=(y_1,y_2,\ldots,y_n)\), edit rastojanje je:

\[
    Edit(i, j) = min
    \begin{cases}
        Edit(i-1, j) + cena\_brisanja \\
        Edit(i, j-1) + cena\_umetanja \\
        Edit(i-i, j-1) + I_{ij} \cdot cena\_zamene \\
    \end{cases}
\]
Vrednost \(I_{ij}\) je 1 ukoliko su \(i\)-ti simbol u \(\overline{X}\) i \(j\)-ti simbol u
\(\overline{Y}\) isti, inace je 0.

Jos jedan pristup koji se koristi za poredjenje objekata sa diskretnim atributima, je metoda
najduzeg zajednickog podniza, ali takvog da ne moraju elementi biti jedan za drugim.
\[
    LCSS(i, j) =
    \begin{cases}
        \emptyset & i = 0 \lor j = 0 \\
        LCSS(i-1, j-1) & i,j > 0 \land x_i = y_i \\
        longest\{LCSS(i, j-1), LCSS(i-1, j)\} & i,j > 0 \land x_i \neq y_i \\
    \end{cases}
\]

\section{Sta je entropija?}
Entropija dogadjaja \(X\) je
\[
    H(X) = - \sum_{i=1}^{n-1}p_i\log_{2}p_i
\]
Broj \(n\) obelezava broj razlicitih klasa podataka koje razmatramo, a \(p_i\) je broj slogova
koji pripadaju klasi \(i\).

\section{Sta su mere na osnovu gustina? Koje se metode najcesce koriste?}
Mere na osnovu gustina odredjuju stepen bliskosti objekata u nekoj oblasti. Najcesce se koristi pri
klasterovanju, i otkrivanju anomalija. Najcesce se koriste
\begin{itemize}
        \item Euklidska gustina --- broj tacaka po jedinici povrsine/zapremine
        \item Gustina verovatnoca --- procena distribucije podataka na osnovu izgleda
        \item Graf zasnovane gustine --- na osnovu povezanosti
\end{itemize}
Primer euklidske gustine bi bila podela regiona na neki broj celioja i definisanje gustine preko
broja tacaka u celijama.

Jos jedan primer bi mogao biti udaljenost od neke centralne tacke.

\end{document}

